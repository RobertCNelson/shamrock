%-------------------------------------------------------------------------
% Preamble
%-------------------------------------------------------------------------
\documentclass[10pt]{article}
\usepackage{palatino}
\usepackage{listings}
\usepackage{color}
\usepackage{textcomp}
%\usepackage{underscore}
\usepackage{hyperref}
\usepackage{amsfonts, amsmath, amssymb}
\usepackage[margin=1.25in]{geometry}
\usepackage{lipsum}
\usepackage{graphicx}
\usepackage{fancyhdr}
\usepackage{parskip}
\usepackage{courier}
\setlength{\parskip}{12pt} 

\pagestyle{fancy}
\setlength\headheight{32pt}

\fancyhead{}
\renewcommand{\headrulewidth}{0pt}
\rhead{\it OpenCL 1.1 Product Version 0.8.2 Readme \\ \color{lightgray}{\rule{6in}{2pt}}}

%\setlength\parindent{0pt}

%-------------------------------------------------------------------------
% Format the footer for each page
%-------------------------------------------------------------------------
\renewcommand{\footrulewidth}{1pt}
\fancyfoot{}
\fancyfoot[L]{}
\fancyfoot[R]{\small{\thepage}}

\definecolor{listinggray}{gray}{0.9}
\definecolor{lbcolor}{rgb}{0.9,0.9,0.9}
\definecolor{lightgray}{rgb}{0.7,0.7,0.7}

\lstset{
        language=c,
        keywordstyle=\bfseries\ttfamily\color[rgb]{0,0,1},
        identifierstyle=\ttfamily,
        commentstyle=\color[rgb]{0.133,0.545,0.133},
        stringstyle=\ttfamily\color[rgb]{0.627,0.126,0.941},
        showstringspaces=false,
        basicstyle=\footnotesize,
        numberstyle=\tiny,
        numbers=left,
        stepnumber=1,
        numbersep=8pt,
        tabsize=2,
        breaklines=true,
        prebreak = \raisebox{0ex}[0ex][0ex]{\ensuremath{\hookleftarrow}},
        breakatwhitespace=false,
        aboveskip={1.5\baselineskip},
        columns=fixed,
        upquote=true,
        extendedchars=true,
        frame=single,
        % backgroundcolor=\color{lbcolor},
}

\linespread{1.1}

\renewcommand{\familydefault}{\sfdefault} 
\usepackage{helvet} 

\begin{document}


%-------------------------------------------------------------------------
% Title
%-------------------------------------------------------------------------
\begin{titlepage}

\vspace {50mm}
{\Large \bfseries OpenCL\textsuperscript{\texttrademark} 1.1 for 66AK2H}\\
\rule{6in}{2pt}

\begin{center}
\vspace {15 mm}
{ \Large \bfseries Product Version 0.8.1 Readme \\[3mm] }
{ \large Publication Date: \today \\ }
\end{center}

\vfill

\includegraphics[scale=0.4]{tiStk2cRgb.pdf}\\
\vspace {2mm}
\rule{6in}{2pt}
{\small
* Product is based on a published Khronos Specification, and is expected to
pass the Khronos Conformance Testing Process. Current conformance status can
be found at www.khronos.org/conformance.
}
\end{titlepage}

\newpage

%-------------------------------------------------------------------------
% TOC
%-------------------------------------------------------------------------
\tableofcontents

\newpage

%-------------------------------------------------------------------------
% Intro
%-------------------------------------------------------------------------
\section{Product Description}

This product is an OpenCL 1.1 implementation.  The OpenCL specification
defines a platform model with a host and compute devices.  For this
implementation the host is a 4-core ARM Cortex-A15 running Linux. There are two 
compute devices: 

\begin{enumerate}
\item The collection of 8 Texas Instruments' C66x DSP cores is
exposed as one virtual accelerator compute device, and
\item The collection of 4 ARM 
Cortex-A15 cores is exposed as a virtual CPU device. (note: The CPU device currently only
supports native kernels)
\end{enumerate}

This OpenCL implementation is typically installed as part of the TI MCSDK-HPC product and environment variables and dependencies are handled automatically as part of the MCSDK-HPC installation. An installation section is still documented here for your knowledge and troubleshooting.  See section \ref{installation} Installation and Dependencies as needed.

\section{OpenCL Documentation}

The OpenCL 1.1 specification and the 1.1 C++ bindings specification from
Khronos are included in the \verb!$(TI_OCL_INSTALL)/doc! sub-directory of this
installation.

Additional OpenCL resources can be found on the web.  Some useful links are provided
below.

\begin{itemize}
\item The OpenCL 1.1 on-line manual pages can be found at:
    \url{http://www.khronos.org/registry/cl/sdk/1.1/docs/man/xhtml}

\item The Khronos OpenCL resources page also has links to good OpenCL
reference material: \url{http://www.khronos.org/opencl/resources}
\end{itemize}




\section{TI OpenCL Extensions}

This OpenCL implementation from Texas Instruments has been extended with a set
of features beyond the OpenCL 1.1 specification.  These features were added in
order to better support the excution of code on the C66 DSP, to enable
existing DSP libraries, and to better map to the 66AK2H hardware platform.

\begin{enumerate}

\item This OpenCL implementation supports the ability to call standard C code from OpenCL C kernels. This includes calling functions from existing C66 DSP libraries, such as the dsplib or mathlib. For examples of this capability please refer to the ccode example in section \ref{ccode} for an example of calling a C function you define, or the dsplib\_fft example in section \ref{dsplibfft} for an example calling a function in a library.

\item Additionally the called standard C code can contain OpenMP pragmas. When using this feature the OpenCL C kernel containing the call to an OpenMP enabled C function must be submitted as a task (not an NDRangeKernel) and it must be submitted to an in-order OpenCL command queue (i.e not defined with the \verb!CL_QUEUE_OUT_OF_ORDER_EXEC_MODE_ENABLE! flag).  In this scenario, OpenCL will dispatch the kernel to one compute unit of the DSP accelerator and the OpenMP runtime will manage distribution of tasks across the compute units. Please see examples vecadd\_openmp in section \ref{vecaddopenmp}, vecadd\_openmp\_t in section \ref{vecaddopenmpt} or openmpbench\_C\_v3 in section \ref{openmpbench}

\item A printf capability has been added for kernels running on the DSP.  OpenCL
   C kernels or standard C code called from an OpenCL C kernel can call
   printf.  The string resulting from the printf will be transmitted to the host ARM and
   displayed using a printf on the ARM side.  This feature can be used to
   assist in debug of your OpenCL kernels.  Note that there is a performance
   penalty in using printf from the DSPs, so it is not a feature that should
   be used when evaluating DSP performance. This feature is not the OpenCL 1.2
   printf which contains additional formatting codes for printing vector
   types.

\item A TI extension for allocating buffers out of MSMC memory has been added. Other
   than the location on the DSP where the buffer will reside, this MSMC
   defined buffer will act as a standard global buffer in all other ways. Example:

\begin{verbatim}
Buffer bufMsmc(context, CL_MEM_READ_ONLY|CL_MEM_USE_MSMC_TI, size);
Buffer bufDdr (context, CL_MEM_READ_ONLY, size); }
\end{verbatim}

The matmpy example in section \ref{matmpy} illustrates the use of MSMC
buffers.  The platform example in section \ref{platform} will query the DSP
device and report the amount of MSMC memory available for OpenCL use.

MSMC stands for Multicore Shared Memory Controller and conaints is an 
on-chip memory shared across all ARM and DSP cores on the chip.

\item The OpenCL C compiler for the C66 DSP supports the C66x standard C compiler set of intrinsic functions, with the exception of those intrinsics that accept or result in a 40 bit value.  Please refer to the C6000 Compiler User's Guide for a list of these intrinsic functions.

\item Additionally these non standard OpenCL C built-in functions are supported:

\begin{verbatim}
uint32_t __core_num     (void);
uint32_t __clock        (void);
uint64_t __clock64      (void);
void     __cycle_delay  (uint64_t cyclesToDelay);
void     __mfence       (void);
\end{verbatim}

    \verb!__core_num! returns [0-7] depending on which DSP core executes the
    function.

    \verb!__clock! return a 32-bit time-stamp value, subtracting two values returned
    by \verb!__clock! gives the number of elapsed DSP cycles between the two.
    This equates to the C66 device's TSCL register.

    \verb!__clock64! return a 64-bit time-stamp value similar to \verb!__clock! but with more
    granularity to avoid potential overflow of a 32 bit counter.
    This equates to the C66 device's TSCH:TSCL register pair.

    \verb!__cycle_delay! takes a specified number of cycles to delay and will busy
    loop for that many cycles (approximately) before returning.

    \verb!__mfence! is a memory fence for the C66x dsp.  Under typical OpenCL use,
    this will not be needed.  However, when incorporating EDMA usage into
    OpenCL C kernels, it may be needed.

    Note: In standard C for C66 a uint32\_t is an unsigned int and a uint64\_t is an
    unsigned long long.  In OpenCL C for C66 a uint32\_t is an unsigned int and
    a uint64\_t is an unsigned long.

\item This OpenCL implementation also supports direct access to the 66AK2H
EDMA system from the DSP and OpenCL C kernels.  A wide range of EDMA
constructs are supported.  These include 1D to 1D, 1D to 2D, 2D to 1D, and chained
transfers.  The edmamgr example in section \ref{edmamgr} shows an example
usage of this feature.

\item An extended memory feature is also supported in this implementation of
OpenCL.  The C66 DSP is a 32-bit architecture and has a limit of 2GB of DDR
that it can access at any given time. The 66AK2H platforms can support up to
8GB of DDR3.  To enable usage of DDRs greater than 2GB, this OpenCL
implementation can use a hardware mapping feature to move windows over the 8GB
DDR into the 32-bit DSP address space.  Movement of these windows will occur
at kernel start boundaries so two sequential kernels dispatched to the DSP
device may actually operate on different 2GB areas within the 8GB DDR.  The
windows are not moved within a kernel.  As a result of this feature, large
buffers may be created and subsequently populated on the ARM side.  However, a
dispatched kernel may not access the entire buffer in one dispatch. Any given
OpenCL Kernel will be limited to a total of 2GB of DDR access.  The example
vecadd\_mpax in section \ref{vecaddmpax} illustrates a process of defining a
large buffer and then defining sub-buffers within the larger buffer and
dispatching multiple OpenCL kernels to the DSP on these sub-buffers,
cumulatively resulting in the entire large buffer being processed.

\end{enumerate}


\newpage
\section{Limitations of this Beta OpenCL Implementation}
This is a Beta version of this product.  It is not a complete OpenCL
implementation and has not successfully completed the Khronos conformance
tests yet.  Specifically, the following features are missing:

\begin{itemize}
\item This OpenCL implementation is not designed to allow multiple OpenCL enabled Linux processes 
to concurrently execute.  Attempting to execute two OpenCL applications
concurrently can leave the DSPs in an unknown state and may require a reboot.
This will be resolved in a future OpenCL release.

\item Dispatching OpenCL C kernels to the CPU device with clEnqueueTask or
clEnqueueNDRangeKernel is not currently supported.  The CPU is still exposed
in the platform because you can submit native code to the CPU using
clEnqueueNativeKernel.  This is illustrated in the ooo and ooo\_map examples \ref{ooo} in
this installation.  All dispatch models are supported for the DSP (accelerator) device.

\item The OpenCL API functions are complete for the DSP device, with the exception of
argument passing to kernels.  The implementation today will support at least
19 arguments and as many as 29 arguments to kernels.  The limit is variable
based on the size of the arguments.  Additionally, structures and vector types are
not supported as arguments yet.  Buffers of vector types are supported.

\item Not all of the OpenCL C built-in functions are available yet. Missing are:

\begin{itemize}
        \item the convert\_type built-ins where saturation or explicit rounding
        modes are specified from section 6.2.3 in the OpenCL 1.1 spec.  The 
        convert\_type built-ins without saturation of explicit rounding are supported.
        \item the atomic built-ins from section 6.11.11 in the OpenCL 1.1 spec
        \item the shuffle and shuffle2 built-ins from section 6.11.12 in the OpenCL 1.1 spec.
        \item the following math built-ins from section 6.11.2 in the OpenCL 1.1 spec:
        cbrt, copysign, erfc, erf, expm1, fdim, fma, fmax, fmin, fract, frexp,
        hypot, ilogb, ldexp, lgamma, lgamma\_r, log1b, logb, maxmag, minmag,
        modf, nan, pown, powr, remainder, remquo, rint, rootn, round,
        sincos, tgamma, trunc
\end{itemize}

\item The half (16bit) floating point format is not supported.

\item Support for images and samplers is optional for non GPU devices per the
OpenCL 1.1 spec and they are not supported by the DSP device in this
implementation.

\item The device info query for frequency on the CPU device is not accurate.  It
currently returns a fixed value of 1.0 Ghz.

\item The OpenCL C built-in functions are not yet optimized for the C66 dsp
device.  

\item The C66x DSP compiler used to compile OpenCL C kernels is not yet optimized for wide vector types.  The resultant code can be inefficient and the compile time
can be long on some code samples using 8 or 16 wide vectors.  It is recommended that vector widths over 4 (2 for doubles and longs) be avoided in this release.

\end{itemize}

\section{Notes}
On-line compilations of OpenCL C code are cached on the system. If you 
  run an OpenCL application that on-line compiles some OpenCL C code,
  the resultant binaries are cached on the system and the next time
  you run the OpenCL application, the compilation step is skipped and
  the cached binaries are used. The caching only uses the OpenCL C
  code and the compile options as a hash, so an example where the
  OpenCL C code is calling a C function in a linked object file or
  library and the object file or library is modified will result in an
  execution of the OpenCL C linked against the older version of the
  object.  In this case you will need to clear the OpenCL C compile
  cache, which can be accomplished with the command 

\verb!rm -f /tmp/opencl*!.

Additionally you can disable this caching behavior by setting the 
\verb!TI_OCL_CACHE_KERNELS_OFF! enviroment variable. See the Enviroment
Variable section \ref{envvar} for more details.

\newpage
\section{Examples}
There are several OpenCL examples shipped with the product.  They are located
in the 'examples' directory within the OpenCL package.

The examples can be cross-compiled in an X86 development environment, or 
compiled native on the ARM A15, depending on the availability of native g++ or
cross-compiled arm-linux-gnueabihf-g++ tool sets.

The example makefiles are setup to cross-compile by default and
assume an ARM cross-compile environment has been installed. If the cross compiler is not installed, execute the following command to install it:

\texttt{sudo apt-get install g++-4.6-arm-linux-gnueabihf}

The environment variables defined in the Installation section
/ref{installation} are required for
correct operation of all examples below.

\subsection{Building and Running the Examples}
From your X86 system: All the examples can be built at one time by invoking 
'make' from the 'examples' directory.

Individual examples can be built by navigating to the desired directory and also
issuing \texttt{make [cross]}.

Once the example is built and the resulting executable is available on the
Hawking EVM file system it can be invoked from within a Hawking EVM xterm
or console window. 


\subsection{Platform Example}\label{platform}
The platform example uses the OpenCL C++ bindings to discover key platform and
device information from the OpenCL implementation and print it to the screen.


\subsection{Simple Example}
This example simply illustrates the minimum steps needed to dispatch a kernel
to a DSP device and read a buffer of data back.


\subsection{Mandelbrot/Mandelbrot\_native Examples}\label{mandelbrot}
The Mandelbrot example is an OpenCL demo that uses OpenCL to generate the
pixels of a Mandelbrot set image.  This example also uses the C++ OpenCL 
binding.  The OpenCL kernels are repeatedly called generating images that are 
zoomed in from the previous image.  This repeats until the zoom factor reaches 
1E15. 

This example illustrates several key OpenCL features:
\begin{itemize}
\item OpenCL queues tied to potentially multiple DSP devices and a 
     dispatch structure that allows the DSPs to cooperatively generate pixel 
     data,
\item The event wait feature of OpenCL,
\item The division of one time setup of OpenCL to the repetitive
     en-queuing of kernels, and
\item The ease in which kernels can be shifted from one device type to another.
\end{itemize}
   

The 'mandelbrot\_native' example is non-OpenCL native implementation (no 
dispatch to the DSPs) that can be used for comparison purposes. It uses OpenMP
for dispatch to each ARM core.

Note: The display of the resulting Mandelbrot images is disabled when this example is cross compiled.  It relies on libsdl capability which is not currently supported on the default Linux file-system included in the MCSDK. 


\subsection{Ccode Example}\label{ccode}
This example illustrates the TI extension to OpenCL that allows OpenCL C code
to call standard C code that has been compiled off-line into an object file or
static library. This mechanism can be used to allow optimized C or C callable
assembly routines to be called from OpenCL C code.  It can also be used to
essentially dispatch a standard C function, by wrapping it with an OpenCL C 
wrapper.  Calling C++ routines from OpenCL C is not yet supported.  You should
also ensure that the standard C function and the call tree resulting from the
standard C function do not allocate device memory, change the cache structure,
or use any resources already being used by the OpenCL runtime. 


\subsection{Matmpy Example}\label{matmpy}
This example performs a 1K x 1K matrix multiply using both OpenCL and a
native ARM OpenMP implementation (GCC libgomp).  The output is the execution
time for each approach ( OpenCL dispatch to the DSP vs. OpenMP dispatching to the 4 ARM A15s ).

This example illustrates the use of local buffers, gloal buffers mapped to MSMC using the TI MSMC entension and
async\_workgroup\_strided\_copy for loading a global buffer into the local
buffer.  The async\_workgroup\_strided\_copy built-in function currently just
performs a memcpy and therefore the data movement and compute do not overlap, but the
structure of the code is illustrative of how this will work in a later release where the async\_workgroup\_strided\_copy will use the device's EDMA capability to offload the data movement from the DSP device.

\subsection{Offline Example}
This example performs a vector addition by pre-compiling an OpenCL kernel into a
device executable file. The OpenCL program reads the file containing the
pre-compiled kernel in and uses it directly.  If you use offline compilation
to generate a .out file containing the OpenCL C program and you subsequently
move the executable, you will either need to move he .out as well or the
OpenCL application will need to specificy a non relative path to the .out
file.

\subsection{Offline\_embed Example}
Similar to the offline example, but instead of compiling the kernel into a
binary file, an embeddable header file is created which can be compiled
directly into the host application.


\subsection{Vecadd\_openmp Example}\label{vecaddopenmp}
This is an OpenCL  + OpenMP example.  OpenCL program is running on the host, 
managing data transfers, and dispatching an OpenCL wrapper kernel to the 
device.  The OpenCL wrapper kernel will use the ccode mode (see ccode example) 
to call the C function that has been compiled with OpenMP options (--omp).  
To facilitate OpenMP mode, the OpenCL wrapper kernel needs to be dispatched as 
an OpenCL Task to an In-Order OpenCL Queue.


\subsection{Vecadd\_openmp\_t Example}\label{vecaddopenmpt}
This is another OpenCL + OpenMP example, similar to vecadd\_openmp. The main
difference w.r.t vecadd\_openmp is that this example uses OpenMP tasks within 
the OpenMP parallel region to distribute computation across the DSP cores.


\subsection{Openmpbench\_c\_v3 Example}\label{openmpbench}
This OpenCL + OpenMP example is derived from EPCC OpenMP microbenchmarks,
v3.0. The syncbench test was modified to dispatch using OpenCL.


\subsection{Vecadd Example}
The same functionality as the vecadd\_openmp example, but expressed fully as an
OpenCL application without OpenMP.  Included for comparison purposes.


\subsection{Vecadd\_mpax Example}\label{vecaddmpax}
The same functionality as the vecadd example, but with extended buffers.  
The example iteratively traverses smaller chunks (sub-buffers) of large buffers.
During each iteration, the smaller chunks are mapped/unmapped for read/write. 
The sub-buffers are then passed to the kernels for processing.  This example
could also be converted to use a pipelined scheme where different iterations of
CPU computation and device computation are overlapped.

NOTE: The size of the buffers in the example (determined by the variable
      'NumElements') is dependent on the available CMEM block size.  
      Currently this example is configured to use buffers sizes for memory
      configurations that can support ~1.5 GB total buffer size. The example
      can be modified to use more (or less) based on the platform memory
      configuration.

\subsection{Vecadd\_mpax\_openmp Example}\label{vecaddmpaxopenmp}
Similar to vecadd\_mpax example, but used OpenMP to perform the parallelization
and the computation.  This example also illustrates that printf() could be used
in OpenMP C code for debugging purpose.

\subsection{Dsplib\_fft Example}\label{dsplibfft}
An example to compute FFT's using a routine from the dsplib library.  This
illustrates Calling a standard C library function from an OpenCL kernel.

\subsection{Ooo and Ooo\_map Examples}\label{ooo}
This Application illustrates several features of OpenCL.
\begin{itemize}
\item Using a combination of In-Order and Out-Of-Order queues
\item Using native kernels on the CPU 
\item Using events to manage dependencies among the tasks to be executed.
A JPEG in this directory illustrates the dependence graph being enforced in the
application using OpenCL events.
\end{itemize}

The Ooo\_Map version additionally illustrates the use of OpenCL map and unmap
operations for accessing shared memory between a host and a device.  The 
Map/Unmap protocol can be used instead of read/write protocol on shared memory
platforms.

\subsection{Null Example}
This application is intended to report the time overhead that OpenCL requires
to submit and dispatch a kernel. A null(empty) kernel is created and
dispatched so that the OpenCL profiling times queried from the OpenCL events reflects only the 
OpenCL overhead
necessary to submit and execute the kernel on the device.  This overhead is for the roundtrip for a single kernel dispath.  In practice, when multiple tasks are being enqueued, this overhead is pipelined with execution and can approach zero.

\subsection{Edmamgr Example}\label{edmamgr}
This application illustrates how to use the edmamgr api to asynchronously move 
data around the DSP memory hierarchy from OpenCL C kernels.  The edmamgr.h header file in this directory enumerates the APIs available from the edmamgr package

\newpage
\section{Environment Variables}\label{envvar}

\begin{tabular}{l p{11cm} }
\textbf{TI\_OCL\_INSTALL} & 
Must be set to the top level directory, where your OpenCL installations resides.\\
\\
\textbf{TI\_OCL\_CGT\_INSTALL} & 
Must be set to the top level directory, where your C66 DSP compiler tools installation resides.\\
\\
\textbf{TI\_OCL\_KEEP\_FILES} & 
When OpenCL C kernels are compiled for DSPs, they are
compiled to a binary .out file in the /tmp sub-directory.
They are then subsequently available for download to the
DSPs for running.  The process of compiling generates
several intermediate files for each source file.  
OpenCL typically removes these temporary files.  However,
it can sometimes be useful to inspect these files.  This
environment variable can be set to instruct the runtime to
leave the temporary files in /tmp.  This can be useful to
inspect the assembly file associated with the out file, to
see how well your code was optimized.\\
\\
\textbf{TI\_OCL\_DEBUG\_KERNEL} & 
The TI IDE and debugger, Code Composer Studio (CCS) is not
required for running OpenCL applications on the 66AK2H,
but if you do have CCS installed and and emulator
connected, you can set this environment
variable to enable assembly statement level debug of your
kernel.  When set, this environment variable will instruct
the OpenCL runtime to pause before dispatch of a kernel.
While paused the runtime will display data to the user
indicating that a kernel dispatch is pending. It will
instruct the user to connect to the board through an
emulator and will display the appropriate break-point
address to use for the start of the kernel code.  Debug
capability has not been a focus for this beta release and
will definitely improve in later releases. Setting up the
emulator and CCS is outside the scope of this Readme.  If
you do have those products, consult the documentation
specific to those products.\\
\\
\textbf{TI\_OCL\_CACHE\_KERNELS\_OFF} & 
This prevents the caching of OpenCL C Kernel
compiles. This can be useful if your have OpenCL C 
kernels that call standard C from object code and you
are actively modifying the standard C code.  The
standard C code is not seen by the caching algorithm
and if you are actively modifying the C code, you
may not pick up your modifications if a cached
version of the OpenCL C kernel is being used.\\
\end{tabular}


\newpage
\section{Installation and Dependencies}\label{installation}

This version of OpenCL is dependent on other packages for proper
   operation. If OpenCL was installed as part of the TI MCSDK-HPC product, it is likely that
   these dependencies are resolved already.  However, they are listed here for
   your knowledge and as a first step in any troubleshooting.

\subsection{CMEM module}
   This module allows for contiguous memory allocation of physical memory on
   the device.

   After executing the Linux command \texttt{lsmod} you should see a cmemk module
   listed.  If it is not listed, then it will need to be installed for proper
   OpenCL operation. Use the Linux modprobe command to install the cmem module.  
   The actual parameters used in the command will differ depending on your 
   DDR3 memory space partition.  The following command and parameter set will 
   likely get you started, however.

\begin{verbatim}
sudo modprobe cmemk phys_start=0x823000000 phys_end=0x880000000
     pools=1x1560281088 phys_start_1=0x0c040000 phys_end_1=0x0c500000
     allowOverlap=1
\end{verbatim}

   Additionally the /dev/cmem device should have appropriate permissions.
   Execute the command \verb!ls -l /dev/cmem! and ensure that read and write 
   permission for user, group and other are set.  If not execute the command

\texttt{sudo chmod 666 /dev/cmem }

   It is also important that a minimum of 1.5GB be reserved from Linux for 
   proper cmem and OpenCL operation.  Setting the uboot environment variable
   mem\_reserve to at least 1536M will ensure this requirement.


\subsection{MPM}
The MPM package contains a client and server that allows the DSPs on the device
to be loaded and run from Linux user space. To ensure that MPM is active,
execute the Linux command 
\texttt{ps -ef | grep mpm} 
and your should see an instance
of mpmsrv running.  If your do not see that, then you can start the service by
running:

\texttt{/usr/sbin/mpmsrv}.

If /usr/sbin/mpmsrv does not exist on your file-system refer to the MCSDK-HPC
setup for installation.

\subsection{MPM-TRANSPORT}
The MPM-TRANSPORT package allows the A15 to read/write the shared memory from Linux user
space. For proper operation, the devices \texttt{/dev/dsp*} need
read/write permission for the user. To ensure that the permissions are
correct, execute the Linux command \texttt{ls -l /dev/dsp*} and ensure that read/write permissions 
are set for user, group and other.  If the permissions are not correct, you
can set them with the command:

\texttt{sudo chmod 666 /dev/dsp*}

\subsection{Required Environment Variables}

The OpenCL package depends on a few environment variables for proper
execution. 

\textbf {Note:} Use EXPORT (bash) or SETENV (csh) to set these as environment variables.

\begin{tabular}{l p{13cm} }
\textbf{TI\_OCL\_INSTALL} & the location of the OpenCL package relative to the ARM A15\\
\textbf{TI\_OCL\_CGT\_INSTALL} & the location of the ARM hosted C66 DSP compiler tools\\
\textbf{LD\_LIBRARY\_PATH} & \verb!$TI_OCL_INSTALL/lib!\\
\textbf{PATH} & \verb!$TI_OCL_INSTALL/bin/arm;$TI_OCL_CGT_INSTALL/bin!\\
\end{tabular}

Additionally if you are cross compiling OpenCL applications for the ARM target from an x86 Ubuntu
machine, on that machine you will also need:

\begin{tabular}{l p{13cm} }
\textbf{TI\_OCL\_INSTALL} & location of the OpenCL package relative to the X86\\
\textbf{TI\_OCL\_CGT\_INSTALL} & location of the x86 hosted C66 DSP compiler tools\\
\textbf{PATH} & \verb!$TI_OCL_INSTALL/bin/x86;$TI_OCL_CGT_INSTALL/bin!\\
\textbf{TARGET\_ROOTDIR} & location of Hawking EVM file system mount point on
the x86 system\\
\end{tabular}

   For platforms were native builds are possible, the cross compiling setup
   is not required, but is still supported for faster compilation.

   The script hawking\_env.sh is included in the examples directory of this
   installation. It illustrates the definition of these environment variables for 
   both the ARM runtime side and a potential X86 cross compiling side.

\subsection{Additional package dependencies}

   There are a few other packages that OpenCL depends on.  These can be
   installed using apt-get on the machine on which you will compile (either
   cross or native) the OpenCL application.  

   \verb!sudo apt-get install binutils-dev mesa-common-dev libboost1.46-dev!\\
   \verb!                     libsqlite3-dev libffi6 zlib1g!

   For platforms that support Simple DirectMedia Layer the package 
   libsdl1.2-dev can also be installed in order to see the mandelbrot display for
   the mandelbrot example. See section \ref{mandelbrot} for information on the
   mandelbrot example.
   
   \verb!sudo apt-get install libsdl1.2-dev!

   Additionally, this OpenCL implementation requires the TI C6000 Compiler
   product for proper execution.  IT can be obtained from
   \url{https://www-a.ti.com/downloads/sds_support/TICodegenerationTools/hpcc6xcgt.htm}.

\end{document}
